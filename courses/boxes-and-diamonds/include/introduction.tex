% !TeX root = ../bd-screen.tex

\chapter{Introduction}

Modal logics are extensions of classical logic by the operators $\Box$
(``box'') and $\Diamond$ (``diamond''), which attach to !!{formula}s.
Intuitively, $\Box$ may be read as ``necessarily'' and $\Diamond$ as
``possibly,'' so $\Box p$ is ``$p$ is necessarily true'' and $\Diamond
p$ is ``$p$ is possibly true.'' As necessity and possibility are
fundamental metaphysical notions, modal logic is obviously of great
philosophical interest. It allows the formalization of metaphysical
principles such as ``$\Box p \lif p$'' (if $p$ is necessary, it is
true) or ``$\Diamond p \lif \Box\Diamond p$'' (if $p$ is possible,
it is necessarily possible).

The operators $\Box$ and $\Diamond$ are \emph{intensional}. This means
that whether $\Box !A$ or $\Diamond !A$ holds does not just depend on
whether $!A$ holds or doesn't.  An operator which is not intensional
is \emph{extensional}. Negation is extensional: $\lnot !A$ holds iff
$!A$ does not; so whether $\lnot !A$ holds only depends on whether
$!A$ holds or doesn't. $\Box$ and $\Diamond$ are not like that:
whether $\Box !A$ or $\Diamond !A$ holds depends also on the meaning
of~$!A$.  While ordinary truth-functional semantics is enough to deal
with extensional operators, intensional operators like $\Box$ and
$\Diamond$ require a different kind of semantics. One such semantics
which takes center stage in this book is relational semantics (also
called possible-worlds semantics or Kripke semantics). 

For the logic which corresponds to the interpretation of $\Box$ as
``necessarily,'' this semantics is relatively simple: instead of
assigning truth values to !!{propositional variable}s, an
interpretation~$\mModel{M}$ assigns a set of ``worlds'' to
them---intuitively, those worlds~$w$ at which $p$ is interpreted as true.
On the basis of such an interpretation, we can define a satisfaction
relation. The definition of this satisfaction relation makes $\Box !A$ satisfied at a world~$w$ iff $!A$ is
satisfied at \emph{all} worlds: $\mSat{M}{\Box !A}[w]$ iff
$\mSat{M}{!A}[v]$ for all worlds~$v$. This corresponds to Leibniz's
idea that what's necessarily true is what's true in every possible world.

``Necessarily'' is not the only way to interpret the $\Box$ operator,
but it is the standard one---``necessarily'' and ``possibly'' are the
so-called \emph{alethic} modalities. Other interpretations read $\Box$
as ``it is known (by some person~$A$) that,'' as ``some person $A$
believes that,'' ``it ought to be the case that,'' or ``it will always
be true that.'' These are epistemic, doxastic, deontic, and temporal
modalities, respectively. Different interpretations of $\Box$ will
make different !!{formula}s logically true, and pronounce different
inferences as valid. For instance, everything necessary and everything
known is true, so $\Box !A \lif !A$ is a logical truth on the alethic
and epistemic interpretations. By contrast, not everything believed
nor everything that ought to be the case actually is the case, so
$\Box !A \lif !A$ is not a logical truth on the doxastic or deontic
interpretations. 

In order to deal with different interpretations of the modal
operators, the semantics is extended by a relation between worlds, the
so-called accessibility relation.  Then $\mSat{M}{\Box !A}[w]$ iff
$\mSat{M}{!A}[v]$ for all worlds~$v$ which are accessible from~$w$.
The resulting semantics is very versatile and powerful, and the basic
idea can be used to provide semantic interpretations for logics based
on other intensional operators. One such logic is intuitionistic
logic, a constructive logic based on L. E. J. Brouwer's branch of
constructive mathematics. Intuitionistic logic is philosophically
interesting for this reason---it plays an important role in
constructive accounts of mathematics---but was also proposed as a logic
superior to classical logic by the influential English philosopher
Michael Dummett in the 20th century. Another application of relational
models is as a semantics for subjunctive, or counterfactual,
conditionals, an approach pioneered by Robert Stalnaker and David K.
Lewis.

This book is an introduction to the syntax, semantics, and proof
theory of intensional logics. It only deals with propositional logics,
although future editions will also treat predicate logics.  The
material is divided into three parts: The first part deals with normal
modal logics. These are logics with the operators $\Box$
and~$\Diamond$. We discuss their syntax, relational models and
semantic notions based on them (such as validity and consequence) and
!!{derivation} systems (both axiomatic systems and tableaux). We establish some
basic results about these logics, such as the soundness and
completeness of the !!{derivation} systems considered, and discuss some
model-theoretic constructions such as filtrations. The second part
deals with intuitionistic logic. Here we discuss natural deduction and
axiomatic derivations, relational and topological semantics, and
soundness and completeness of the !!{derivation} systems. The third part deals
with the Lewis-Stalnaker semantics of counterfactual conditionals. The
appendices discuss some ideas and results from set theory and the
theory of relations that's crucial to the relational semantics, and
review syntax, semantics, and !!{derivation} theory of classical propositional
logic.
